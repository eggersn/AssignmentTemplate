\renewcommand{\theenumi}{(\alph{enumi})}
% (a), (b), etc. nummeriert werden.
\renewcommand{\labelenumi}{\text{\theenumi}}

\newcounter{sheetnr}
\newcounter{exnum}

% Befehl für die Aufgabentitel
\newcommand{\exercise}[1]{\section*{Aufgabe \theexnum\stepcounter{exnum}: #1}} % Befehl für Aufgabentitel

\newcommand{\cformula}[1]{\smallskip\centerline{#1}\smallskip}
\newcommand{\inst}[2]{\mathrel{\stackrel{\makebox[0pt]{\mbox{\normalfont\tiny{#1}}}}{{#2}}}}
\newcommand{\bcup}[0]{\mathlarger\bigcup\limits}
\newcommand{\bcap}[0]{\mathlarger\bigcap\limits}
\newcommand{\hdelta}[0]{\hat{\delta}}
\renewcommand{\|}[0]{\:|\:}
\newtheorem*{ntheorem}{Theorem}
\definecolor{DarkGreen}{RGB}{0,100,0}

\LetLtxMacro{\temp}{\phi}
\LetLtxMacro{\phi}{\varphi}
\LetLtxMacro{\varphi}{\temp}
\LetLtxMacro{\temp}{\epsilon}
\LetLtxMacro{\epsilon}{\varepsilon}
\LetLtxMacro{\varepsilon}{\temp}

\lstdefinestyle{pseudo}{
  backgroundcolor=\color{white},   % choose the background color; you must add \usepackage{color} or \usepackage{xcolor}; should come as last argument
  basicstyle=\footnotesize,        % the size of the fonts that are used for the code
  breaklines=true,                 % sets automatic line breaking
  commentstyle=\color{DarkGreen},    % comment style
  escapeinside={(*}{*)},           % if you want to add LaTeX within your code
  frame=single,	                   % adds a frame around the code
  extendedchars=true,              % lets you use non-ASCII characters; for 8-bits encodings only, does not work with UTF-8
  keepspaces=true,                 % keeps spaces in text, useful for keeping indentation of code (possibly needs columns=flexible)
  keywordstyle=\bfseries,       % keyword style
  morekeywords={function, var, begin, for, each, do,
   od, fi, if, not, end, accept, reject, then, break, true,
   false, while},    		   % if you want to add more keywords to the set
  numbers=left,                    % where to put the line-numbers; possible values are (none, left, right)
  numbersep=5pt,                   % how far the line-numbers are from the code
  rulecolor=\color{black},         % if not set, the frame-color may be changed on line-breaks within not-black text (e.g. comments (green here))
  showspaces=false,                % show spaces everywhere adding particular underscores; it overrides 'showstringspaces'
  showstringspaces=false,          % underline spaces within strings only
  showtabs=false,	               % show tabs within strings adding particular underscores
  tabsize=2,	                   % sets default tabsize to 2 spaces
  stringstyle=\color{mymauve},     % string literal style
  morecomment=[l]{//},
    literate=	{Ö}{{\"O}}1
			{Ä}{{\"A}}1
			{Ü}{{\"U}}1
			{ß}{{\ss}}2
			{ü}{{\"u}}1
			{ä}{{\"a}}1
			{ö}{{\"o}}1,
	morecomment=[l]{//}
}

\lstdefinestyle{while}{
  backgroundcolor=\color{white},   % choose the background color; you must add \usepackage{color} or \usepackage{xcolor}; should come as last argument
  basicstyle=\footnotesize,        % the size of the fonts that are used for the code
  breaklines=true,                 % sets automatic line breaking
  commentstyle=\color{DarkGreen},    % comment style
  escapeinside={(*}{*)},           % if you want to add LaTeX within your code
  frame=single,	                   % adds a frame around the code
  extendedchars=true,              % lets you use non-ASCII characters; for 8-bits encodings only, does not work with UTF-8
  keepspaces=true,                 % keeps spaces in text, useful for keeping indentation of code (possibly needs columns=flexible)
  keywordstyle=\color{blue},       % keyword style
  morekeywords={WHILE, LOOP, END, DO,
  IF, ENDIF, THEN, ELSE},    		   % if you want to add more keywords to the set
  numbers=left,                    % where to put the line-numbers; possible values are (none, left, right)
  numbersep=5pt,                   % how far the line-numbers are from the code
  rulecolor=\color{black},         % if not set, the frame-color may be changed on line-breaks within not-black text (e.g. comments (green here))
  showspaces=false,                % show spaces everywhere adding particular underscores; it overrides 'showstringspaces'
  showstringspaces=false,          % underline spaces within strings only
  showtabs=false,	               % show tabs within strings adding particular underscores
  tabsize=2,	                   % sets default tabsize to 2 spaces
  stringstyle=\color{mymauve},     % string literal style
  morecomment=[l]{//},
    literate=	{Ö}{{\"O}}1
			{Ä}{{\"A}}1
			{Ü}{{\"U}}1
			{ß}{{\ss}}2
			{ü}{{\"u}}1
			{ä}{{\"a}}1
			{ö}{{\"o}}1,
	morecomment=[l]{//}
}

\lstdefinestyle{goto}{
  backgroundcolor=\color{white},   % choose the background color; you must add \usepackage{color} or \usepackage{xcolor}; should come as last argument
  basicstyle=\footnotesize,        % the size of the fonts that are used for the code
  breaklines=true,                 % sets automatic line breaking
  commentstyle=\color{DarkGreen},    % comment style
  escapeinside={(*}{*)},           % if you want to add LaTeX within your code
  frame=single,	                   % adds a frame around the code
  extendedchars=true,              % lets you use non-ASCII characters; for 8-bits encodings only, does not work with UTF-8
  keepspaces=true,                 % keeps spaces in text, useful for keeping indentation of code (possibly needs columns=flexible)
  keywordstyle=\color{blue},       % keyword style
  morekeywords={GOTO, IF, THEN, HALT},    		   % if you want to add more keywords to the set
  numbers=left,                    % where to put the line-numbers; possible values are (none, left, right)
  numbersep=5pt,                   % how far the line-numbers are from the code
  rulecolor=\color{black},         % if not set, the frame-color may be changed on line-breaks within not-black text (e.g. comments (green here))
  showspaces=false,                % show spaces everywhere adding particular underscores; it overrides 'showstringspaces'
  showstringspaces=false,          % underline spaces within strings only
  showtabs=false,	               % show tabs within strings adding particular underscores
  tabsize=2,	                   % sets default tabsize to 2 spaces
  stringstyle=\color{mymauve},     % string literal style
  morecomment=[l]{//},
    literate=	{Ö}{{\"O}}1
			{Ä}{{\"A}}1
			{Ü}{{\"U}}1
			{ß}{{\ss}}2
			{ü}{{\"u}}1
			{ä}{{\"a}}1
			{ö}{{\"o}}1,
	morecomment=[l]{//}
}

\ohead{
%Authors
}
\chead{\Large Übungsblatt \thesheetnr}
\ihead{
%title \\
%tornus \\
%group
}

